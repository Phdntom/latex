\documentclass{exam}
\usepackage[margin=1.0in]{geometry}
\usepackage[export]{adjustbox}
\usepackage{amsmath,amsthm,amssymb, graphicx, multicol, array}

\begin{document}
Useful Equations:
\vspace{5mm}

$\frac{\partial^2f}{\partial x^2}=\frac{1}{v^2}\frac{\partial^2f}{\partial t^2}$ \hspace{\stretch{1}}
$f(x,t) = f(x \pm vt)$
\hspace{\stretch{1}}
$v=\sqrt{\frac{F}{\mu}}$
\hspace{\stretch{1}}
$power = \frac{1}{2} \mu \omega^2 A^2 v$
\hspace{\stretch{1}}
$v = \lambda f = \frac{\omega}{k}$
\vspace{\stretch{1}}

$y(x,t) = A sin\left[\frac{2\pi x}{\lambda}-\frac{2\pi t}{T}+\phi\right]$
\hspace{\stretch{1}}
$\omega = 2\pi f = \frac{2\pi}{T}$
\hspace{\stretch{1}}
$k=\frac{2\pi}{\lambda}$
\hspace{\stretch{1}}
$f_{beat}=|f_a-f_b|$
\vspace{\stretch{1}}

$\Delta \Phi = \frac{2\pi \Delta x}{\lambda}+\frac{2\pi \Delta t}{T}+\Delta \phi$
\hspace{\stretch{1}}
Point Source: $I = \frac{Power}{4\pi r^2}$
\hspace{\stretch{1}}
Two Source: $I = I_0 cos^2(\Delta \Phi/2)$

\vspace{\stretch{1}}



string with both ends fixed or open pipe: $f_n=n\frac{v}{2L}(n=1,2,3,4...)$
\hspace{\stretch{1}}
sound in air: $v=344m/s$
\vspace{\stretch{1}}

stopped pipe: $f_n = n\frac{v}{4L} (n=1, 3, 5, ...)$
\hspace{\stretch{1}}
$f_L=f_S\left( \frac{v+v_L}{v+v_S}\right)$ positive velocity from listener to source
\vspace{\stretch{1}}

$\beta = 10dB log_{10}\frac{I}{I_0}$
\hspace{2mm}
$I_0=10^{-12}W/m^2$
\hspace{\stretch{1}}
$2t = m\lambda$
\hspace{4mm}
$2t = (m+\frac{1}{2})\lambda$
\hspace{\stretch{1}}
$c = 3\times10^8m/s$
\vspace{\stretch{1}}

interference maxima: $d sin\theta = m\lambda$
\hspace{\stretch{1}}
$\lambda_i = \lambda_{vac}/n_i$
\hspace{\stretch{1}}
visible light: 400nm - 700nm(violet to red)
\vspace{\stretch{1}}

diffraction: $I = I_0 \left(\frac{sin[\beta/2]}{\beta/2}  \right)^2$ \hspace{2mm} $\beta = 2\pi\frac{a sin\theta}{\lambda}$
\hspace{\stretch{1}}
circular aperture: $sin\theta \approx \theta=1.22\frac{\lambda}{D}$


\vspace{\stretch{1}}

diffraction minima: $a sin\theta=m\lambda (m=\pm1, \pm2, ...)$
\hspace{\stretch{1}}
$n_1 sin\theta_1=n_2 sin\theta_2$
\vspace{\stretch{1}}

sign conventions for geometrical optics

object distance, $S$ - positive on side of incoming light

image distance, $S'$ - positive on side of outgoing light

curvature, $R$ - positive when center of circle on side of outgoing light

focal length, $f$ - positive on side of outgoing light

lateral magnification, $M$ - positive when image is upright
\vspace{\stretch{1}}


plane reflector: $S'=-S$
\hspace{\stretch{1}}
plane refractor: $S'=-\frac{n_2}{n_1}S$
\hspace{\stretch{1}}
$M=-\frac{S'}{S}$
\vspace{\stretch{1}}

spherical reflector: $\frac{1}{S} + \frac{1}{S'}=\frac{1}{f}=\frac{2}{R}$
\hspace{\stretch{1}}
$M=-\frac{S'}{S}$
\hspace{\stretch{1}}
$M_\theta=\theta'/\theta$
\vspace{\stretch{1}}

spherical refractor: $\frac{n_1}{S} + \frac{n_2}{S'}=\frac{n_2}{f}=\frac{n_2-n_1}{R}$
\hspace{\stretch{1}}
$f=\frac{n_2R}{n_2-n_1}$
\hspace{\stretch{1}}
$M=-\frac{n_1}{n_2}\frac{S'}{S}$
\vspace{\stretch{1}}

thin lenses: $\frac{1}{f}=(n-1)\left[\frac{1}{R_1}-\frac{1}{R_2}\right]$
\hspace{\stretch{1}}
$\frac{1}{f}=\frac{1}{S}+\frac{1}{S'}$
\hspace{\stretch{1}}
$H=\frac{dQ}{dT}=kA\frac{dT}{dx}$
\vspace{\stretch{1}}

$\Delta L = \alpha L_0 \Delta T$
\hspace{\stretch{1}}
$\Delta V = \beta V_0 \Delta T$
\hspace{\stretch{1}}
$\beta=3\alpha$
\hspace{\stretch{1}}
$Q=m C \Delta T$
\hspace{\stretch{1}}
$Q=\pm mL$
\vspace{\stretch{1}}


water: $C_{liquid}=4190\frac{J}{kgK}$
\hspace{\stretch{1}}
$C_{ice}=2100\frac{J}{kgK}$
\hspace{\stretch{1}}
$C_{steam}=2000\frac{J}{kgK}$
\hspace{\stretch{1}}
$L_f=3.34\times10^5 \frac{J}{kg}$
\hspace{\stretch{1}}
$L_v=2.26\times10^6 \frac{J}{kg}$

\newpage
\centering(blank page)
\newpage




\title{Midterm 2 - Physics 9B-A - Spring 2023}
\author{}
\date{}
\maketitle





\vspace{20mm}
\makebox[0.75\textwidth]{Name :\enspace\hrulefill}

\vspace{10mm}
\makebox[0.75\textwidth]{Student ID:\enspace\hrulefill}
\newpage

\begin{questions}

\question [30] You study light with a cheap grating. In fact, it barely qualifies as a grating because it has only four slits. Accordingly, there is noticeable non-zero intensity between the points of completely constructive interference. At the zero-order point of constructive interference, the phase difference between adjacent slits is of course zero. At the first-order maximum the difference is $2\pi$.

\begin{parts}
    \part There are three points in between the zero-order and first-order maxima for which the interference will be completely destructive. For what phase difference $\phi$ do each of these minima correspond? Sketch a phasor diagram for each.
\vspace{\stretch{2}}
    \part Calling the amplitude from a single slit $E$, what is the total amplitude at a point where the phase difference between adjacent slits is $\pi/4$?
\vspace{\stretch{3}}
    \part What is the ratio of the intensity at the point in part (b) to the intensity at the central point of complete constructive interference? (Don't work too hard on this one.)
\vspace{\stretch{1}}
\end{parts}
\clearpage


\question [15] A satellite has a camera that gathers visible light through a circular aperture of $20cm$ diameter. Assume its resolution is limited by diffraction alone, and it needs to resolve details on the ground as close together as 50cm. What is the maximum altitude above Earth's surface at which it can orbit? (You will have to make a reasonable assumption about the wavelength.)
\vspace{\stretch{1}}
\clearpage

\question [20] The diagram shows two light rays incident from air on the left end of a long thin piece of plastic fiber.

\includegraphics[width=0.3\textwidth, right]{plast_fiber.png}


\begin{parts}
    \part Which ray (Ray 1 or Ray 2) would be more likely to continue on through the fiber all the way to the far right end (not pictured), endlessly bouncing off the top, then bottom, never escaping? Briefly explain.
\vspace{\stretch{1}}
    \part It is possible to choose a refractive index for the plastic so that light rays incident at the left end at all angles would never escape. What minimum refractive index would necessary to accomplish this?
\end{parts}

\vspace{\stretch{2}}
\clearpage

\question[15]
Santa observes his reflection in a silvered spherical ornament $0.750m$ away. The diameter of the ornament is $7.2cm$. Assuming his height is $1.6m$, where is the image of Santa formed by the ornament? How tall is the image and is it upright or inverted?

\vspace{\stretch{1}}
\clearpage



\question [20] 
Using a combination of two lenses you study a little animal from a safe distance. The animal/object is only $5cm$ tall and is $3.00m$ to the left of a converging lens with a focal length of $1.00m$. From the right, you look through the second lens, a diverging lens with focal length $-0.40m$, that is $1.00m$ to the right of the first lens. Compute the location and the size of the final image of the animal.
\clearpage


\question [25] 
A $0.010kg$ ice cube at $-30^\circ C$ is placed in a $0.30kg$ copper cup initially at $50^\circ C$. The ice cube melts completely. The specific heat of copper is $385\frac{J}{kgK}$.
\begin{parts}
    \part Find the final temperature of the system, assuming no heat exchange with the surroundings.
\vspace{\stretch{2}}

    \part Is it possible to put multiple $-30^\circ C$ ice cubes into the cup so that none of the ice is melted in the final state?  If so, find the minimum number of ice cubes needed.
\vspace{\stretch{3}}

\end{parts}
\clearpage

\question[20]
A wall consists of a $3.0cm$ thick outer layer of wood and a $4.0cm$ inner layer of Styrofoam. This wood has a thermal conductivity $0.120\frac{W}{m K}$ and Styrofoam has thermal conductivity of $0.010\frac{W}{m K}$. The outside temperature is a constant $5.0^\circ C$ and the inside temperature is a constant $20^\circ C$.
\begin{parts}
    \part How does the heat conducted per unit time through the wood compare to that through the Styrofoam? Briefly Explain.
\vspace{\stretch{1}}
    \part What is the temperature at the boundary between the wood and the Styrofoam?
\vspace{\stretch{4}}

    
\end{parts}



\end{questions}

\end{document}
