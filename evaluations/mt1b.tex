\documentclass{exam}
\usepackage[margin=1.0in]{geometry}
\usepackage[export]{adjustbox}
\usepackage{amsmath,amsthm,amssymb, graphicx, multicol, array}

\begin{document}
Useful Equations:
\vspace{5mm}

$\frac{\partial^2f}{\partial x^2}=\frac{1}{v^2}\frac{\partial^2f}{\partial t^2}$ \hspace{\stretch{1}}
$f(x,t) = f(x - vt)$
\hspace{\stretch{1}}
$v=\sqrt{\frac{F}{\mu}}$
\hspace{\stretch{1}}
$power = \frac{1}{2} \mu \omega^2 A^2 v$
\hspace{\stretch{1}}
$v = \lambda f = \frac{\omega}{k}$
\vspace{\stretch{1}}

$y(x,t) = A sin\left[\frac{2\pi x}{\lambda}-\frac{2\pi t}{T}+\phi\right]$
\hspace{\stretch{1}}
$\omega = 2\pi f = \frac{2\pi}{T}$
\hspace{\stretch{1}}
$k=\frac{2\pi}{\lambda}$
\hspace{\stretch{1}}
$f_{beat}=|f_a-f_b|$
\vspace{\stretch{1}}

$\Delta \Phi = \frac{2\pi \Delta x}{\lambda}+\frac{2\pi \Delta t}{T}+\Delta \phi$
\hspace{\stretch{1}}
Point Source: $I = \frac{Power}{4\pi r^2}$
\hspace{\stretch{1}}
Two Source: $I = I_0 cos^2(\Delta \Phi/2)$

\vspace{\stretch{1}}



string with both ends fixed or open pipe: $f_n=n\frac{v}{2L}(n=1,2,3,4...)$
\hspace{\stretch{1}}
sound in air: $v=344m/s$
\vspace{\stretch{1}}

stopped pipe: $f_n = n\frac{v}{4L} (n=1, 3, 5, ...)$
\hspace{\stretch{1}}
$f_L=f_S\left( \frac{v+v_L}{v+v_S}\right)$ positive velocity from listener to source
\vspace{\stretch{1}}

$\beta = 10dB log_{10}\frac{I}{I_0}$
\hspace{2mm}
$I_0=10^{-12}W/m^2$
\hspace{\stretch{1}}
$2t = m\lambda$
\hspace{4mm}
$2t = (m+\frac{1}{2})\lambda$
\hspace{\stretch{1}}
$c = 3\times10^8m/s$

\vspace{\stretch{1}}

double slit maxima: $\Delta x = d sin\theta = m\lambda$
\hspace{\stretch{1}}
$\lambda_i = \lambda_{vac}/n_i$
\hspace{\stretch{1}}
visible light: 400nm - 700nm(violet to red)

\newpage
Page Intentionally Left Blank
\newpage




\title{Physics 9B-B - Midterm 1, Spring 2023}
\author{}
\date{}
\maketitle





\vspace{20mm}
\makebox[0.75\textwidth]{Name :\enspace\hrulefill}

\vspace{10mm}
\makebox[0.75\textwidth]{Student ID:\enspace\hrulefill}
\newpage

\begin{questions}

\question [30] A wave traveling on a string is described at t=0 by $y(x,0) = (0.001)e^{-3x^2}$. It is moving in the $+x$ direction at speed 4. (Assume distances are in meters, time in seconds).

\includegraphics[width=0.30\textwidth, right]{wave.png}

\begin{parts}
    \part At what point along the x-axis is the acceleration of the medium (i.e. segments of the string) zero? Partial credit will be given if you describe qualitatively where these points might be.
\vspace{\stretch{2}}
    \part At the points with zero acceleration, in what direction is the velocity of the medium?
\vspace{\stretch{1}}
    \part Determine $y(x,t)$
\vspace{\stretch{1}}
\end{parts}
\clearpage


\question [10] A long heavy rope hangs from the ceiling. You give it a quick shake at the bottom, initiating a wave pulse that moves upward along the rope. As it moves up the rope, does the pulse speed up, slow down, or move at constant speed? Explain.
\vspace{\stretch{1}}

\question [15] Sitting in adjacent desks 1.0m apart are Anna, Bob, and Carl (Bob in the middle). Carl is talking, the sound spreading out in all directions, while Anna and Bob are studying quietly. If Bob hears a sound intensity level from Carl of 61dB, what sound intensity level does Anna hear?
\vspace{\stretch{1}}
\clearpage


\question [25] A string of 5.0g mass is stretched under a tension of 200N between two fixed supports 100cm apart. It is plucked and produces a musical sound, a combination of many standing waves that can coexist on it, though each higher harmonic contributes considerably less sound than the one below it.
\begin{parts}
    \part You notice that the fundamental, responsible for most of the sound you hear, causes a nearby pipe to resonate and produce its fundamental standing wave mode. The pipe is closed at one end only. What is its length?
\vspace{\stretch{1}}
    \part You now hold your finger lightly against the string at a point 40cm from one end. The sound coming from the string drops off quite a bit, but you still hear something. What's the dominant frequency you hear?
\vspace{\stretch{1}}
\end{parts}
\clearpage


\question [15] You are on a motor bike heading north along a street at 10m/s. You sound your bike's 90Hz horn. A truck is ahead of you, traveling toward you blowing its 90Hz horn. You detect 6 beats per second between the sound directly from your horn and the sound from the truck's horn. Find the velocity of the truck (specify direction with North or South). If there are multiple answers, you only need to provide one.
\clearpage


\question [20] You pull a wire ring out of soapy water and hold it in front of you as you would a handheld mirror. The refractive index of this soapy water is 1.42. White light strikes the thin film inside the ring and reflects back to you. You see horizontal bands of bright colors, different colors at different vertical levels.

\includegraphics[width=0.2\textwidth, right]{soap_ring.png}
\begin{parts}
    \part What kind of interference would you expect in the region at the top, where the soap film is very thin, and more importantly, why?
\vspace{\stretch{1}}
    \part Starting from the top, bright bands occur for orange light with wavelength in air 600nm (the first two shown in the image). What is the thickness of the soap film where the 2nd orange band is?
\vspace{\stretch{2}}
\end{parts}
\clearpage


\question [20] Light of 480nm wavelength strikes a barrier with two narrow slits separated by 0.30mm. On a screen 6.0m away you observe a familiar intensity pattern, in which the intensity is zero at point P in the figure. (Any angles measured to the screen from the slits can be considered small for this problem).

\includegraphics[width=0.3\textwidth, right]{2slit.png}

\begin{parts}
    \part How much farther does light from the bottom slit have to travel than light from the top slit to reach point P?
    \vspace{\stretch{2}}
    \part If the intensity at the central maximum is $I_0$, what is the intensity at a point one-third of the distance from the central maximum to point P?
    \vspace{\stretch{3}}

\end{parts}

\end{questions}

\end{document}
