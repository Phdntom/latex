\documentclass[10pt]{article}
 \usepackage[margin=0.5in]{geometry} 
\usepackage{amsmath,amsthm,amssymb, graphicx, multicol, array}
  
\newenvironment{problem}[2][Problem]{\begin{trivlist}
\item[\hskip \labelsep {\bfseries #1}\hskip \labelsep {\bfseries #2.}]}{\end{trivlist}}

\begin{document}
%---------------
%---------------
 \title{Problem Set \# 4 Solutions}
\date{}
\maketitle

\begin{problem}{1}
Reflected light remains in the same medium as the incident light. The bending of light toward the normal comes from the change in speed which ultimately depends on the index of refraction in the medium which can be wavelength dependent. Without a medium change in the case of reflected light, there is no dependence on the wavelength and the light reflects via the Law of Reflection, independent of the medium change.

\end{problem}


 \begin{problem}{2}
\item For both parts, the "outside" is either in the air $n=1.0$ or water $n=1.33$ and the light starts inside the glass with $n=1.53$. Using Snell's law, aka the Law of Refraction: $n_1 sin\theta_1 = n_2 sin\theta_2$ where we will take $n_1$ to be glass and $n_2$ to be air/water, using the convention that $n_1$ denotes the medium where the light is incident on the boundary. Finally, we want the critical angle where no light enters the outside medium (air/water), viz. we want total internal reflection and that corresponds to $\theta_2=\pi/2$. We simply plug in in both case (a) and (b):

\item a) $(1.53) sin\theta_c= (1.0) sin(\pi/2)$ or $\theta_c=0.71rad=41^\circ$

\item b) $(1.53) sin\theta_c= (1.33) sin(\pi/2)$ or $\theta_c=1.05rad=60^\circ$

\item
Hopefully these two answers make sense. The lower angle for air means that total internal reflection starts happening for smaller angles than for water. This is explained by the larger difference in index of refraction for air vs glass than water vs glass.
\end{problem}


\begin{problem}{3}
There are two boundaries to consider in this problem. We can write Snell's at both boundaries and then enforce total internal reflection at the vertical face. That will tell us the critical angle on the face, which we can relate to the transmitted angle at the top using geometry. 
\item Law of refraction at top: $n_1sin\theta_1=n_2sin\theta_2$.
\item Law of refraction at side: $n_2sin\theta_3=n_1sin\theta_4$.
\item Total internal reflection needs to happen at the side boundary, so $\theta_4=90^\circ$. We use this to get $\theta_3=\theta_c=44^\circ$
\item Inside the block, with some geometry, $\theta_3=90-\theta_2$ so $\theta_2=46^\circ$. At this point it's worth noting we want the maximum for $\theta_1$ at the top, and that corresponds to the largest $\theta_2$ which in turn corresponds to the smallest $\theta_3$. When we calculate the critical angle, we are finding the minimum angle. All angles larger than $44^\circ$ will have T.I.R. so we can take the minimum one as that gives the largest angle at the top.
\item With $\theta_2=46^\circ$ in hand, we can solve: $sin\theta_1=(1.38)/(1.0)sin46^\circ$ or $\theta_1=83^\circ$

\end{problem}






%-------------
%-------------
\end{document}
