\documentclass[10pt]{article}
 \usepackage[margin=0.5in]{geometry} 
\usepackage{amsmath,amsthm,amssymb, graphicx, multicol, array}
  
\newenvironment{problem}[2][Problem]{\begin{trivlist}
\item[\hskip \labelsep {\bfseries #1}\hskip \labelsep {\bfseries #2.}]}{\end{trivlist}}

\begin{document}
%---------------
%---------------
 \title{Discussion - Cyclic Processes}
\date{}
\maketitle

\begin{problem}{1}
An ideal monatomic gas in a cylinder with a moveable piston starts at a volume of 1L, a pressure of 1atm and a
temperature of 300K.
\item ---Step 1: While the piston is held in place, the gas is heated to 800K.
\item ---Step 2: It then expands, doing work on external machinery, quickly enough that heat has essentially no time to escape through the walls of the cylinder. This expansion ends when the pressure has dropped back to 1atm.
\item ---Step 3: Then, the gas cools back to its original volume of 1L while the piston is allowed to move freely, keeping the pressure at 1atm.
\item (a) Sketch the process on a PV diagram.
\item (b) What is the pressure at the end of Step 1?
\item (c) What are the volume and the temperature at the end of Step 2?
\item (d) Find the change in the internal energy of the gas in each of the steps, and the total.
\item (e) Find the work done by the gas in each step, and the total. [Use the 1st Law whenever possible to save effort.]
\item (f) Find the heat added to the gas in each step, and the total.
\item (g) How does the net work done by the gas in this “closed cycle” compare to the heat added in Step 1? As a ratio, this is the efficiency.
\end{problem}








%-------------
%-------------
\end{document}
