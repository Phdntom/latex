\documentclass[10pt]{article}
 \usepackage[margin=0.5in]{geometry} 
\usepackage{amsmath,amsthm,amssymb, graphicx, multicol, array}
  
\newenvironment{problem}[2][Problem]{\begin{trivlist}
\item[\hskip \labelsep {\bfseries #1}\hskip \labelsep {\bfseries #2.}]}{\end{trivlist}}

\begin{document}
%---------------
%---------------
 \title{Discussion - Waves on a String Notes}
\date{}
\maketitle

\begin{problem}{1}

\item (a) Any sketch resembling a bell curve will suffice.
\item (b) Need 1st and 2nd derivatives:$\frac{\partial y}{\partial x} = (0.05) \frac{-2x}{(1+x^2)^2}$ and $\frac{\partial^2y}{\partial x^2}=\frac{(-2)(1+x^2)^2 +(2x)(2)(1+x^2)(2x)}{(1+x^2)^4}$. Setting the 2nd derivative to zero, $-2(1+x^2) + 8x^2=0$ and $x=\pm \frac{1}{\sqrt{3}}$
\item (c) If the spatial 2nd derivative is zero, so is the temporal 2nd derivative.
\item (d) The middle is the max height. The velocity will be max where its derivative (the acceleration) is zero, which are the points of inflection from part (b).
\item (e) $y(x, t)=(0.05m)\frac{1}{1+(x-3t)^2}$.
\item (f) $\frac{\partial y}{ \partial t}=\frac{6(x-3t)}{(1+(x-3t)^2)^2}$.
\item (g) Setting (f) to zero, $x=3t$ is a solution. This corresponds to the top of the pulse which is always a maximum as the wave travels.
\item (h) $x=\pm \frac{1}{\sqrt{3}}+3t$. The answer from part (d) is modified by the propagation of the wave. The points of inflection are where $v_y$ is maximum, and the POI moves to the right just like all other parts of the wave.


\end{problem}








%-------------
%-------------
\end{document}
