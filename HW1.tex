\documentclass[10pt]{article}
 \usepackage[margin=0.5in]{geometry} 
\usepackage{amsmath,amsthm,amssymb, graphicx, multicol, array}
  
\newenvironment{problem}[2][Problem]{\begin{trivlist}
\item[\hskip \labelsep {\bfseries #1}\hskip \labelsep {\bfseries #2.}]}{\end{trivlist}}

\begin{document}
%---------------
%---------------
 \title{Homework \# 1}
\date{}
\maketitle
 
\begin{problem}{1}
Two waves are created on the same string.
\item
a) List the parameters that characterize a wave pulse on a string. Are there additional ones for a periodic wave? What are they?
\item
b) Using subscripts for the parameters you listed in (a) for each wave, compare the same quantity (i.e. $A_1$ vs $A_2$) and decide if the parameters can take different values if the string is the same.
\item
c) For two periodic waves, redo (b) with the two waves now having the same frequency on the same string.
\end{problem}

\begin{problem}{2}
Sine and Cosine functions are valid solutions to the wave equation (given the correct arguments) and are simple cases of periodic waves. In both cases, the function is positive as often as it is negative. Why isn't the total power of a sinusoidal wave zero?
\end{problem}

\begin{problem}{3}
While bored in lab, you create a makeshift assembly of two strings into a much longer one for "learning purposes only". Due to budget cuts and lack of consistent supplies, the two smaller pieces are from leftover scraps and thus have different densities $\mu_1$ and $\mu_2$. Naturally, you affix the string to some supports and start sending waves down the string to study physical principles. What properties will be the same on each side of the connection and which will be different? Clearly explain for both wave pulses and periodic waves.
\end{problem}

\begin{problem}{4}
After witnessing some rogue students studying physics by assembling pieces of string, your instructor seizes the opportunity for even more learning.  You are given three pieces of string, each of length $L$, that are joined together. The densities are provided relative that of the first piece: $\mu_1$, $\mu_2=5\mu_1$, $\mu_3=\mu_1/2$. You place the assembled string of length $3L$ under tension $F$.
\item
a) How long does it take for a transverse wave to move the length of the string?
\item
b) If you were to rearrange the pieces in a different order, would your answer in (a) change? Explain.
\end{problem}

\begin{problem}{5}
A wave is described by the wave function,
\begin{align}
y(x, t) = (8.0mm) sin 2 \pi \left( \frac{x}{32cm} - \frac{t}{0.05s}\right) \nonumber
\end{align}
\item
a) Identify the amplitude, period, and wavelength. Calling positive to the right, what direction is the wave moving?
\item
b) What is the wave speed?
\item
c) Graph the wave when $t=0.40s$.
\item
d) At $t=0.40s$, what is the velocity of the string at $x=8cm$?
\item
e) When $t=0$, what is the velocity of the string at $x=8cm$?
\item
f) Suppose this wave function describes a wave on a string of density $\mu=5g/m$ and compute the average power transmitted by the wave.
\end{problem}



%-------------
%-------------
\end{document}
