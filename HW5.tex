\documentclass[10pt]{article}
 \usepackage[margin=0.5in]{geometry} 
\usepackage{amsmath,amsthm,amssymb, graphicx, multicol, array}
  
\newenvironment{problem}[2][Problem]{\begin{trivlist}
\item[\hskip \labelsep {\bfseries #1}\hskip \labelsep {\bfseries #2.}]}{\end{trivlist}}

\begin{document}
%---------------
%---------------
 \title{Homework \# 5}
\date{}
\maketitle

\begin{problem}{1} Spherical Reflectors\\
A person looks at his reflection in the concave side of a shiny spoon. Is it right side up or inverted? Does it matter how far his face is from the spoon? What if he looks in the convex side?
\end{problem}

\begin{problem}{2} Thin Lenses\\
You take a lens and mask it so that light can pass through only the bottom half of the lens. How does the image formed by the masked lens compare to the image formed before masking?
\end{problem}

\begin{problem}{3} Plane Reflectors\\
The image of a tree just covers the length of a plane mirror 4cm tall when the mirror is held 35cm from the eye. The tree is 28m from the mirror. What is its height?
\end{problem}

\begin{problem}{4} Spherical Reflectors\\
A coin is placed next to the convex side of a thin spherical glass shell having radius of curvature 18.0cm. Reflection from the surface of the shell forms an image of the 1.5cm tall coin that is 6.0cm behind the glass shell. Where is the coin located? Determine the size, orientation, and nature (real or virtual) of the image.
\end{problem}

\begin{problem}{5} Spherical Refractors\\
A small fish is at the center of a spherical fish bowl with a diamter of 28cm.
\item a) Find the apparent position and magnification of the fish to an observer outside the bowl. Ignore any effects of the thin walls of the bowl and treat the refracting medium as the water in the bowl.
\item b) A friend advised the owner of the bowl to keep it out of direct light to avoid blinding the fish in case the fish swam into the focal point of the bowl. Is the focal point actually within the bowl?
\end{problem}

\begin{problem}{6} Thin Lenses\\
A converging lens has a focal length of 14.0cm. For an object to the left of the lens, consider two cases where the distance is 18.0cm and 7.0cm. In each case, draw a principal ray diagram and determine:
\item a) The image position,
\item b) The magnification,
\item c) Whether the image is real or virtual,
\item d) Whether the image is upright or inverted.
\end{problem}

\begin{problem}{7} Thin Lenses\\
An object to the left a lens produces an image on a screen $30cm$ to the right of the lens. When the lens is moved $4cm$ to the right, the screen must be moved $4cm$ to the left to refocus the image. What is the focal length of the lens?
\end{problem}

\begin{problem}{8} Lens Maker's Equation\\
A converging lens is made with two curved surfaces with radii of curvature $R_1=+12cm$ and $R_2=+60cm$ using a material with refractive index $n=1.6$. A small object that is $8mm$ tall when measured perpendicular the optical axis is placed a distance $30cm$ to the left of the lens. A second identical lens is placed $50cm$ to the right of the first lens. Find the position and size (magnification) of the final image. Specify if the final image is upright or inverted with respect to the original object.
\end{problem}

\begin{problem}{9} Angular Magnification\\
A telescope is constructed from two lenses with focal lengths 95cm and 15cm, the 95 cm lens used as the objective lens. Both the object being viewed and the final image are at infinity.
\item a) Find the angular magnification.
\item b) Find the height of the image formed by the objective of a building 60m tall, 3km away.
\item c) What is the angular size of the final image as viewed as viewed by an eye very close to the eyepiece?
\end{problem}







%-------------
%-------------
\end{document}
