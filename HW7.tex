\documentclass[10pt]{article}
 \usepackage[margin=0.5in]{geometry} 
\usepackage{amsmath,amsthm,amssymb, graphicx, multicol, array}
  
\newenvironment{problem}[2][Problem]{\begin{trivlist}
\item[\hskip \labelsep {\bfseries #1}\hskip \labelsep {\bfseries #2.}]}{\end{trivlist}}

\begin{document}
%---------------
%---------------
 \title{Problem Set \# 7}
\date{}
\maketitle

\begin{problem}{1} Kinetic Theory of Gases

\item a) What is the average translational kinetic energy of a molecule of an ideal gas at $T=27^\circ C$?
\item b) What is the total translational energy of 1 mole of the molecules in (a)?
\item c) If oxygen gas is at this temperature, what is the root-mean-square (RMS) speed an oxygen molecule?
\end{problem}

\begin{problem}{2} Ideal Gas
\item A metal tank with volume $3.10L$ will burst if the absolute pressure of the gas it contains exceeds $100atm$.
\item a) If 11.0 mol of an ideal gas is put into the tank at $T=23.0^\circ C$, what is the maximum temperature the gas can be heated to before the tank exceeds its pressure tolerance. Neglect the thermal expansion of the tank.
\item b) Based on your answer above, was it reasonable to neglect the thermal expansion? Explain.
\end{problem}

\begin{problem}{3} Thermodynamic Processes\\
A flexible balloon contains 0.350 mol of hydrogen sulfide gas ($H_2S$). Initially the gas has a volume of $7.0 \times 10^3 cm^3$ and a temperature of $27^\circ C$. The gas first expands isobarically until the volume doubles. Then it expands adiabatically until the temperature returns to its initial state. Assume that the hydrogen sulfide can be treated as an ideal gas.
\item a) Draw a PV-diagram of the process.
\item b) What is the total heat supplied to the gas during the process?
\item c) Calculate the change in internal energy of the gas.
\item d) What is the total work done by the gas?
\item e) What is the final volume?
\end{problem}

\begin{problem}{4} Cyclic Process\\
A cylinder contains oxygen at $P=2.00atm$. The volume is $4.0L$, and the temperature is $300K$. Assume an ideal gas as the oxygen undergoes the following processes:
\item ---(i) Heated isobarically from initial state (1) to state (2) at $T=450K$;
\item ---(ii) Cooled at constant volume to state (3) at $T=250K$;
\item ---(iii) Compressed at constant temperature to $4.00L$ in state (4);
\item ---(iv) Heated at constant volume to $300K$, taking the system back to state (1).

\item a) Show these four processes in a PV-diagram, giving numerical values to P and V.
\item b) Calculate Q and W for each of the four processes.
\item c) Find the net work done by the oxygen gas.
\item d) What is the efficiency of the device as a heat engine?
\item e) Compare the efficiency in (d) to that of a Carnot-cycle engine operating between the same minimum and maximum temperatures of $250K$ and $450K$.
\end{problem}

\newpage

\begin{problem}{5} Entropy and Heat\\
A student adds heat to a block of ice, $0.350kg$, at $0^\circ C$ until it's all melted.
\item a) What is the change in entropy of the water?
\item b) The source of the heat, the heat reservoir, is massive body at a temperature of $25.0^\circ C$. What is the change in entropy of the reservoir?
\item c) Now find the total change in entropy of the combined system of the block plus reservoir.
\end{problem}

\begin{problem}{6} Multiplicity and Entropy\\
A box is partitioned in half with 500 molecules of nitrogen gas on the left and 100 molecules of oxygen gas on the right and with both sides at the same temperature. The partition is punctured allowing a free expansion of the gases until equilibrium is reached.
\item a) On average, how many molucules of each type will be in either half of the box?
\item b) What is the change in entropy of the system from initial configuration to equilibrium?
\item c) What is the probability of finding the molecules distributed as they were initially, with 500 nitrogen on the left and 100 oxygen on the right?
\end{problem}

%-------------
%-------------
\end{document}
