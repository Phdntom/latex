\documentclass[10pt]{article}
 \usepackage[margin=0.5in]{geometry} 
\usepackage{amsmath,amsthm,amssymb, graphicx, multicol, array}
  
\newenvironment{problem}[2][Problem]{\begin{trivlist}
\item[\hskip \labelsep {\bfseries #1}\hskip \labelsep {\bfseries #2.}]}{\end{trivlist}}

\begin{document}
%---------------
%---------------
 \title{Problem Set \# 3}
\date{}
\maketitle

\begin{problem}{1}
It is often useful for radio antennas to radiate most of their energy in particular directions rather than uniformly. This can be done with rows or just pairs of antennas. Consider two such antennas $400m$ apart, operating at $1500kHz$ (this is at the top of the AM band) that are in phase. For distances much larger than $400m$, find all directions with the greatest intensity. What is the drawback of doing this?
\end{problem}
 


\begin{problem}{2}
You are studying a nearby planet using an infrared telescope with diameter, $D=1.0m$. (Use $10\mu m$ for infrared radiation). What is the smallest details you could resolve on the planet if you know the planet is a distance $2.0\times10^{17}m$ away?
\end{problem}

\begin{problem}{3}
Two microscope slides with refractive index $n_g=1.55$ are placed on top of each other but with a small flake accidentally between them on one end. The result is that the two slides are actually only in contact on one end, with a water solution filling the remaining space between them (Water solution has $n_w=1.33$). When you view the slides from above, you see a bright line of violet light ($400nm$) that is $1.15mm$ from the ends in contact but no other visible light between the violet line and the ends in contact.
\item a) How far from the line of contact will you see a similar bright line for green light (550nm)?
\item b) How far from the line of contact will you see a second line of bright violet light?
\item c) No calculations here. If you replaced the water solution with air, what, if anything, changes?
\end{problem}






\begin{problem}{4}
Light of wavelength 585nm falls on a single slit 0.0666mm wide.
\item a) On a very large and distant screen, how many totally dark fringes (total cancellation) will there be? Do not calculate all the angles but instead find the largest value of "m", the order of the fringe.
\item b) Given the fringe most distant from the central maxima from part (a), find the angle to the screen at which the fringe occurs.
\end{problem}

\begin{problem}{5}
Consider the inference pattern formed by two parallel slits of width $a$ and separation $d$, in which $d=3a$. Light of wavelength $\lambda$ illuminates the slits.
\item a) Ignoring diffraction, at what angles from the central maximum will the next four maxima in the two-slit interference pattern happen? You can leave your answer in terms of $d$ and $\lambda$.
\item b) Accounting for diffraction now, and taking the central maximum intensity as $I_0$, what is the intensity at each of the angles from part (a)?
\item c) Are all of the maximum from the two slit interference visible? Explain.
\end{problem}




%-------------
%-------------
\end{document}