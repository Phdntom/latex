\documentclass[10pt]{article}
 \usepackage[margin=0.5in]{geometry} 
\usepackage{amsmath,amsthm,amssymb, graphicx, multicol, array}
  
\newenvironment{problem}[2][Problem]{\begin{trivlist}
\item[\hskip \labelsep {\bfseries #1}\hskip \labelsep {\bfseries #2.}]}{\end{trivlist}}

\begin{document}
%---------------
%---------------
 \title{Midterm Practice}
\date{}
\maketitle

\begin{problem}{1}
When sound travels from air into water, does the wavelength change? Explain.
\end{problem}

\begin{problem}{2}
A transverse sine wave with amplitude 3.0mm and a wavelength of 2.50m travels from left to right along a horizontal string with speed 40m/s. Take the origin at the left of the undisturbed string. At time t=0 the left end of the string is moving upward.
\item a) What is the wave function $y(x,t)$ that describes the wave?
\item b) Find the transverse velocity of a particle 3.30m to the right of the origin at time t=0.0625 seconds.
\end{problem}

\begin{problem}{3}
A wave traveling on a string is described by $y(x,0)=A\frac{1}{1+x^2}$. It moves in the $+x$ direction at speed 3.
\item a) At what points along the x-axis is the acceleration of the medium zero? You will get partial credit for describing where these points might be.
\item b) In what direction is the velocity of the medium at the points you found in (a)?
\item c) Determine $y(x,t)$
\end{problem}
 
\begin{problem}{4}
You and a friend have a long stretchy string (possibly made of a rubbery material). You stretch the string to be reasonable taught and you send a wave down the string. As you do this your friend begins to run, stretching the string more as he runs and the wave travels down the string. Explain what happens to the wave speed of the wave you created as it "chases" your friend. Does it speed up, slow down or move at constant speed?
\end{problem}

\begin{problem}{5}
An organ pipe has two successive harmonics with frequencies 660Hz and 924Hz. The speed of sound in air is 344m/s.
\item a) Is this an open (both ends), or a stopped (closed at one end) pipe? Explain.
\item b) What is the length of the pipe?
\end{problem}

\begin{problem}{6}
At an airport the maximum allowable sound intensity level is 95dB as measured by a microphone at the end of a 1600m long runway. A certain airliner produced a sound intensity level of 98dB on the ground when it flies over at an altitude of 100m. On takeoff, this airliner rolls 1200m along the runway before leaving the ground, at which point it climbs at a $17^\circ$ angle. Does this airliner exceed the maximum allowable sound intensity level measured at the microphone as it flies over? Ignore all reflection effects form the ground.
\end{problem}

\begin{problem}{7}
A double slit apparatus is setup and the fringes displayed on a screen. Now, the entire setup is submerged in water. How does the fringe pattern change?
\end{problem}

\begin{problem}{8}
When viewing a piece of art that is behind glass, there is often glare, making it difficult to see the art clearly. One solution to glare is to coat the outer surface of the glass with a thin film to cancel part of the glare.
\item a) If the glass has a refractive index of 1.62 and you use a coating with refractive index 2.62, what is the minimum thickness to cancel out light of wavelength 505nm?
\item b) If this coat is too thin to last over time, what other thicknesses of the coating would work? Find the three thinnest ones.
\end{problem}

\begin{problem}{9}
Coherent light containing two wavelengths, 660nm(red) and 470nm(blue) passes through two slits separated by a distance 0.30mm. The interference pattern is observed on a screen 8.0m from the slits. On the screen, what is the distance between the first order red fringe and first order blue fringe?
\end{problem}



%-------------
%-------------
\end{document}