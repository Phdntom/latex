\documentclass[10pt]{article}
 \usepackage[margin=0.5in]{geometry} 
\usepackage{amsmath,amsthm,amssymb, graphicx, multicol, array}
  
\newenvironment{problem}[2][Problem]{\begin{trivlist}
\item[\hskip \labelsep {\bfseries #1}\hskip \labelsep {\bfseries #2.}]}{\end{trivlist}}

\begin{document}
%---------------
%---------------
 \title{Problem Set \# 4}
\date{}
\maketitle

\begin{problem}{1}
When light is incident on an interface between two materials, the angle of the refracted ray depends on the wavelength, but the angle of the reflected ray does not. Why is this?
\end{problem}


 \begin{problem}{2}
A beam of light is traveling inside a solid glass cube that has an index of refraction 1.53. It strikes the surface of the cube from the inside.
\item a) If the cube is in air, at what minimum angle with the normal inside the glass will this light not enter the air at this surface?
\item b) What is the new minimum angle if you submerge the cube in water?
\end{problem}


\begin{problem}{3}
A ray of light is incident in air on the top of a clear solid block with index of refraction $n=1.38$. The light refracts through the top and strikes a vertical face of the block. What is the largest angle of incidence at the top, such that total internal reflection will occur when light hits the vertical face?
\end{problem}

\begin{problem}{4}
A converging lens has a focal length of 14.0cm. For an object to the left of the lens, consider two cases where the distance is 18.0cm and 7.0cm. In each case, draw a principal ray diagram and determine:
\item a) The image position,
\item b) The magnification,
\item c) Whether the image is real or virtual,
\item d) Whether the image is upright or inverted.
\end{problem}

\begin{problem}{5}
An object to the left a lens produces an image on a screen $30cm$ to the right of the lens. When the lens is moved $4cm$ to the right, the screen must be moved $4cm$ to the left to refocus the image. What is the focal length of the lens?
\end{problem}

\begin{problem}{6}
A converging lens is made with two curved surfaces with radii of curvature $R_1=+12cm$ and $R_2=+60cm$ using a material with refractive index $n=1.6$. A small object that is $8mm$ tall when measured perpendicular the optical axis is placed a distance $30cm$ to the left of the lens. A second identical lens is placed $50cm$ to the right of the first lens. Find the position and size (magnification) of the final image. Specify if the final image is upright or inverted with respect to the original object.
\end{problem}




%-------------
%-------------
\end{document}
