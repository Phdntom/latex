\documentclass[10pt]{article}
\usepackage[margin=0.5in]{geometry} 
\usepackage{amsmath,amsthm,amssymb, graphicx, multicol, array}
  
\newenvironment{problem}[2][Problem]{\begin{trivlist}
\item[\hskip \labelsep {\bfseries #1}\hskip \labelsep {\bfseries #2.}]}{\end{trivlist}}

\begin{document}
%---------------
%---------------
 \title{Problem Set \# 6 Solutions}
\date{}
\author{}
\maketitle

\begin{problem}{1} Thermal Expansion\\
\item Use $\Delta L = \alpha L_0 \Delta T$ for both cases:
\item In Death Valley - $\Delta L = (2.6\times10^{-5}K^{-1}) (0.0190m) (48^\circ C - 20^\circ C)=1.3832\times10^{-5}m$. $L=L_0+\Delta L=1.901cm$
\item In Greenland - $\Delta L = (2.6\times10^{-5}K^{-1}) (0.0190m) (-53^\circ C - 20^\circ C)=-3.6062\times10^{-5}m$. $L=L_0+\Delta L= 1.896cm$

\end{problem}



\begin{problem}{2} Heat Capacity\\
The unknown material sample is solid and presumed to stay solid so no phase change. The copper calorimeter is also safely solid in this temperature range. For water, the final temp is known to be $26.1^\circ C$ and we start at $19.1^\circ C$ so no phase changes.
\item $-Q_{sample} = Q_{calorimeter} + Q_{water}$
\item $-m_{sample}C_{sample}\Delta T_{sample} = m_{cal}C_{cal}\Delta T_{cal} + m_w C_w \Delta T_{w}$
\item $-(0.085kg)C_{sample}(26.1^\circ C - 100^\circ C) = (0.015kg)(385J/kgK)(26.1^\circ C - 19.0^\circ C) + (0.200)(4180J/kgK)(26.1^\circ C - 19.0^\circ C)$
\item $C_{sample} = 951J/kgK$
\end{problem}



\begin{problem}{3} Temperature Dependent Molar Heat Capacity\\
\item a) Starting with the provided hint, $dQ=nCdT=n(kT^3/\Theta^3)dT$, we then integrate from $10.0K$ to $40.0K$. The idea is that at each value of $T$ the heat capacity is different, so we use integration to "add up" the contributions to the total heat from the variable heat capacity at each infinitesemal temperature $dT$.

\[
Q = \int  \,dQ = \int_{10}^{40} nk \frac{T^3}{\Theta^3} \,dT = \frac{nk}{4\Theta^3}T^4\Big|_{10}^{40}=\frac{(1.5mol)(1940\frac{J}{molK})}{4(281K)^3}\left[(40K)^4-(10K)^4\right]=83.6J
\]
\item b) We want to know the average molar heat capacity in a range of temperature, so our function is $C = kT^3/\Theta^3$ and the formula for average of any function is given below:
\[
\langle C \rangle = \frac{1}{T_2-T_1}\int_{T_1}^{T_2} k \frac{T^3}{\Theta^3} \,dT
\]
This is nearly the same integral as (a) except without the n=1.5mol and with the inverse temperature difference:
\[
\langle C \rangle = \frac{1}{40K-10K}\frac{(1940\frac{J}{molK})}{4(281K)^3}\left[(40K)^4-(10K)^4\right]=1.86\frac{J}{molK}
\]
\item c) For this we can evaluate at $T=40K$, $C=\frac{kT^3}{\Theta^3}=1940\frac{J}{molK}\frac{(40K)^3}{(281K)^3}=5.60\frac{J}{molK}$

\end{problem}



\begin{problem}{4} Heat Capacity and Phase Changes\\
The starting temperature is $0^\circ C$ because there is both water and ice. Since we are adding ice with a temperature $-15^\circ C$, and there is ice left at equilibrium, the final temperature of the system is either zero or below zero.

\item To figure out if the system is below zero or not, let's do a "test" calculation. When the ice is added, its temperature will increase. This energy to raise the temperature has to come from somewhere, and it's from the intial water/ice system at $0^\circ$. The system can't lower its temperature until all of the water is turned to ice. So that's our test, suppose we froze all of the initial water and used that energy loss to heat up the added ice. This will tell us if we have water remaining or not, and also if the final temperature is zero or below zero.
\item 
$m_{water}L=m_{added}C_{ice}(0 - (-15^\circ C))\rightarrow{} m_{added}=\frac{(1.75kg)(334000J/kg)}{(2100J/kgK)(15^\circ C)}=18.56kg$
\item
That is certainly more than the final amount of ice we are told is present at the end, so this means not all of the water is going to turn to ice before the system reaches equilibrium, still at $T=0^\circ C$. So now we can proceed to setup our final calculation.
\item $m_{w}L=m_{added}C_{ice}(0 - (-15^\circ C))$
\item This looks the same as before but now $m_w$ is the amount of water that actually freezes, not all of the initial water, and thus is unknown for now. So now we have two unknown masses, but we know the total mass, $m_t=m_{added}+m_w+m_i$ where $m_i$ is the initial amount of ice in the system and will be included as part of the final amount of ice, which is known. We can now solve this system of two equations for both, $m_w$ and $m_{added}$.
\item 
$m_w=0.036kg$ and $m_{added}=0.382kg$
With these values known, it's easy to replay and check the calculation.
\item $m_wL=12024J$
\item $m_{added}C\Delta T=12033J$
\item $0.450kg+0.036kg+0.382=0.868kg$
\end{problem}



\begin{problem}{5} Heat Transfer and Phase Changes\\
$H=\frac{dQ}{dt}=-kA\frac{dT}{dx}$
\item The heat melts 8.50g of ice in 10 min, so the heat to melt the ice is $Q=mL=(0.0085kg)(334000J/kg)=2839J$ and that takes $10min=600s$, so the heat transfer $H=2839J/600s=4.73W$.
\item We can rearrange our equation for heat transfer to get he thermal conductivity, $k=HL/(A\Delta T)$ where we have used the fact that the temperature gradient is linear to replace $dT/dx=\Delta T/L$. Plugging in, $k=\frac{(4.73W)(0.6m)}{(1.25\times10^{-4}m^2)(100^\circ C)}=227\frac{W}{mK}.$
\end{problem}



\begin{problem}{6} Heat Transfer\\
$H=\frac{dQ}{dt}=kA\frac{dT}{dx}$. We will want to work with $H/A$ as no area is provided for the wall but like $H$, $H/A$ has to be the same everywhere in the wall to avoid build up of energy in the wall. Letting $T$ denote the temperature of the wood/Styrofoam boundary, we write the heat transfer across both regions and set them equal, first going from inside to the boundary and then boundary to the outside:
\item $H/A=k_s\Delta T_s/L_s = k_w \Delta T_w/L_w$.
\item $k_s(T-19^\circ C)/L_s = k_w(-10^\circ C-T)/L_w$.
\item $T(k_sL_w+k_wL_s)=-(10^\circ C)k_wL_s + (19^\circ C) k_sL_w$
\item $T = -5.78^\circ C$
\item 
This is often counter-intuitive. Styrofoam is a worse conductor of heat, so the temperature drop is greatest inside the Styrofoam, $\Delta T_s = -5.78^\circ C -  19^\circ C=-24.78^\circ C$. In the wood, $\Delta T_w = -10^\circ C -  -5.78^\circ C=-4.22^\circ C$. And the total drop is the sum, $-29^\circ C$. Because H is the same everywhere, it is the same in Styrofoam as wood, but H is proportional to $k\Delta T$ so if the $k$ is lower for Styrofoam, $\Delta T$ must be larger to make up the difference, and vice versa for wood.


\end{problem}








%-------------
%-------------
\end{document}
